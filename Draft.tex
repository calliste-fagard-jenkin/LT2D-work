\documentclass[11pt]{article}
\usepackage{multicol,graphicx,float,tikz,tikz-qtree,scrextend,apacite,amsbsy,inputenc,amsmath}
\usepackage[margin=0.7 in]{geometry}
\title{A Method for Analysing Heterogeneous Data with 2D Detection Function Distance Sampling}
\date{January 2018}
\author{Calliste Fagard-Jenkin}
\inputencoding{utf8}
\begin{document}
\pagenumbering{gobble}
\maketitle
\begin{center}
\includegraphics[scale=1]{Logo}
\end{center}


\newpage
\emph{I  certify  that  this  project  report  has  been  written  by  me,  is  a  record  of  work  carried  out  by  me,  and  is essentially different from work undertaken for any other purpose or assessment.}

\tableofcontents
\pagenumbering{arabic}
\newpage
\begin{multicols}{2}

\newpage

\section{Abstract}
This project focuses on line-transect distance sampling; a hybrid method which uses both design and model based inference to produce estimates on the sizes of animal populations. An extension of distance sampling by \cite{Borchers} allows for the use of 2D detection functions, describing the probability of detection of an animal in both forward and perpendicular directions. This methodology produces estimates of abundance which are far less biased when compared to estimates produced by conventional distance sampling, when the animals in the population of interest display responsive movement. The raison d'être of this project is to further extend the code-base of the \cite{Borchers} methodology's R package (called LT2D) to include the possibility of covariate inclusion to cope with data that are heterogeneous with respect to detection probability, and to provide a number of illustrative analyses on a range of simulated datasets to demonstrate the improved functionality.

\subsection{Intended readership and summary of objectives}
This project focuses heavily on computation within statistical ecology. Good knowledge of undergraduate statistics is preferable in order to understand the contents. Ideally the reader also has an understanding of animal abundance estimation (specifically distance sampling) and has experience analysing and handling datasets which include covariates. These prerequisites can be summarised by a level of statistical knowledge contained within the MT3507 and MT3508 modules, with knowledge of content within the MT5751 module being ideal.

The intended demographic of readers is researchers in ecological statistics, as well as ecologists who will benefit from reading the latter sections of the project in order to learn how to use the new LT2D software to perform 2D detection function distance sampling analyses on data which display responsive movement with respect to observers, as well as heterogeneity with respect to detection probability. 

This project includes sections which encapsulate computer-theoretic as well as purely statistical content, in order to explain issues of software design, implementation and consideration of user interaction. This is done both to give the reader an in-depth appreciation of how the project achieved its desired outcomes and also to reflect the distribution of work hours the project required. The code-base being modified as part of this project is substantial and complicated, and therefore any implementation of added functionality requires a significant time commitment.

succinctly, the aim of this project is to implement covariate inclusion into the hazard detection functions in the LT2D package resulting from \cite{Borchers}, to add a bootstrap measure of uncertainty on estimates of abundance into the package, to modify goodness of fit and plotting functions to handle covariate-including models and to provide example analyses of simulated data to provide ecologists with knowledge on how to analyse their own data with the new package.

\section{Introduction}
\subsection{Context and background}


\subsubsection{What is distance sampling?}

Distance sampling is a methodology within ecological statistics which allows us to produce unbiased estimates of animal population sizes even when the detection of any individual in the survey area is uncertain. First, a quick description of its simpler alternative, strip sampling (in which detection of individuals is certain), will be given in order to illustrate the fundamental principles of greatest importance. 

Consider a survey area within which the population of interest is closed (there is no emigration or immigration of individuals). We further assume that births and deaths are considered to occur at a negligible rate throughout the duration of the survey. In this way, the survey is considered to be a snapshot of the population at some given moment. In our example, we choose a square survey area, which we subdivide into five strips of regular width. We must further assume that animals are uniformly distributed throughout the survey area, in both the perpendicular and the horizontal axis (in fact, we must simply assume that the distribution of animals with respect to the transect is known. Uniformity arises as a natural consequence of independently distributed transect lines in conventional distance sampling. In the case of point transect distance sampling, the assumed distribution is triangular). The below figure illustrates uniformly distributed animals within a survey area.

\begin{figure}[H]
\centering
\includegraphics[scale=0.5]{StripSampling}
\caption{Example distribution of animals in a square survey area divided into 5 equal strips}
\end{figure}

Given this rather simplistic design, obtaining an estimate of abundance in the survey area ($\hat{N}$) is extremely simple, so long as we are able to sample all of the individuals within whole strips. As an example we consider strips 2 and 4 (from the left) as the randomly selected strips which form our survey sample. Strip 2 contains 25 individuals, and strip 4 contains 15, giving us a total of $n=40$ observed individuals. If we denote the total survey area by $A$ and the sampled area by $a$, then we observe that our survey effort of $\frac{a}{A}=\frac{2}{5}$. Hence a reasonable intuitive estimator for the abundance in the whole survey area is

\begin{equation}
\hat{N}=\frac{n}{\frac{a}{A}}=\frac{40}{0.4}=100
\end{equation}

which is a simple rescaling of the number of observed animals which accounts for the portion of the survey area which has remained unseen. This example forms a basic overview of an easy methodology we can use when detection of every individual is guaranteed. We would now like to find a way of relaxing the unrealistic assumption of perfect detection. If we assume all animals in the strip have some fixed probability $p$ of being detected, then we can modify our estimator to account for the animals we expect we have missed:

\begin{equation}
\hat{N}=\frac{n}{\frac{a}{A}p}
\end{equation} 


This again is nothing more than rescaling the estimate by an appropriate constant, to account for animals that we believe were present, but not detected by any observers. 


The real hard work begins when we start to ask ourselves how we can estimate this value of average detection probability, which turns out not to be so simple. However, distance sampling comes to the rescue!  In line transect distance sampling, one or multiple observers travel along a transect line which has been pre-determined by the experimental design. Observers look to the left and right of the line as they travel along its path (either on foot, by plane, boat, car etc) and record the perpendicular distance at which each observed animal was detected. Once we pool together all these perpendicular distances for all detected animals, we expect to see a histogram similar to figure 2, which clearly displays that the frequency of detection drops as perpendicular distance increases. This trend has an obvious explanation, owing to the fact that animals become harder and harder to see the farther away they find themselves from the observer(s).

\begin{figure}[H]
\includegraphics[scale=0.5]{DistanceHist}
\caption{Histogram of perpendicular distances of detected animals}
\end{figure}


A model-based approach which involves the fitting of a detection function, allows us to estimate the way in which perpendicular distance affects detection probability. Figure 3 illustrates this by overlaying a (scaled) half-normal detection function over simulated detection data. Over the course of an analysis, multiple detection functions will be fitted to the data and the most appropriate one selected. In the case of the Distance package written and maintained by the Centre for Research into Ecological and Environmental Modelling, Saint Andrews, this is done by the numerical computation of maximum likelihood estimates of the detection function parameters, and model selection is determined by goodness-of-fit tests as well as information criteria such as Akaike's Information Criterion.


If we momentarily return to the case of perfect detection, we notice that we would expect to see a perfectly flat histogram of perpendicular detection distances (as a result of uniform distribution of animals about the transect line). In this case the bars of our histogram would perfectly cover both the shaded, and un-shaded areas of the large rectangle containing Figure 3. Hence, the ratio of the area under the fitted curve of our detections with respect to the area of this 'perfect detection rectangle' gives us some indication of the average detection probability of the population of interest. In practice, the area of the 'perfect detection rectangle' is extremely easy to calculate (the height of the rectangle is the frequency of detections at perpendicular distance 0, since distance sampling assumes perfect detection on the transect. The width of the rectangle is the perpendicular truncation distance $w$ of the study). The area covered by our own detections is seen as the integral of the detection function over an interval from $0$ to $w$. In order to obtain the sampled area ($a$, in the previous example) in the case of a line transect distance sampling survey, we simply consider $a=2Lw$, where $L$ is the total length of all covered transect lines and $w$ is the previously mentioned perpendicular truncation distance. Thus, with these tools at our disposal, we can modify (2) to obtain:

\begin{equation}
\hat{N}=\frac{n}{\frac{2Lw}{A}\hat{p}}
\end{equation}


where $\hat{p}$ is the estimate of average detection probability we obtain by integrating the detection function and considering the area of the 'perfect detection rectangle'. It should be noted that this description of distance sampling is extremely rudimentary, and is far more intuitive than it is technical. For a deeper understanding of the likelihood functions involved and the design-based, and model-based aspects of the method, reference should be made to both classic and modern literature, such as \cite{EAB} and \cite{DS2015}.

\begin{figure}[H]
\centering
\includegraphics[width=6.38cm]{DistHistDetec}
\caption{Histogram of detection distances with overlayed detection function}
\end{figure}

\subsubsection{The significance of 2D detection functions}

One of the assumptions of distance sampling is that the distribution of animals about the transect line is known. In the previous case of line-transects, it was uniform. Unfortunately, there are many situations in which this assumption is strongly violated, leading to significant bias in estimates of animal abundance and density. One reason for this could be poor survey design, whereby insufficient randomisation of transect line locations leads to trends in the distribution of animals between them. There is no effective analytical solution for poor survey design, and so this case is not considered here. A more interesting cause for this phenomenon is responsive movement of animals in the population of interest. Animals can be scared by observers and hence move away, causing a dip in observed detection frequencies on and close to the transect line, as well a spike in observed detection frequency some relatively small distance away from the transect line. This behaviour is typically referred to as avoidance, and is extremely common in primates, with gibbons being so notoriously prone to the behaviour that they are the subject of regular conventions between ecologists and statisticians. Conversely, animals may also display attraction, whereby an attraction to the transect line causes a spike in detections on the transect line and a dip further out. Dolphins typically display this behaviour when data is collected via shipboard surveys, where the wake of the vessel draws them in.

By considering time-to-detection models, and observing that when an observer is moving along the transect at a constant speed, the forward distance of a detected animal is directly proportional to the time to detection (assuming the animal is stationary), \cite{Borchers} formulate a method by which the relationship between distance and detection probability can be estimated in two-dimensions. This extra information provided in the forward distances of detections also allows the estimation of the perpendicular distributions about the transect line of the detected animals. Once this distribution is known, the issues of responsive movement are largely remedied. It should be noted that the vast majority of studies which collect data which are suitable for the fitting of 2D detection functions often do so by measuring the radial distance to the animal, as well as the angle of incidence of the animal to the observer. Thus the values of perpendicular and forward difference are easily calculated using basic trigonometry. Many of the implementations and intricacies involved with handling edge cases arise directly from this convention and the ways it affects the data and their collection.

\subsection{Previous contributions}

From June - August 2017 I undertook work on the code-base as a summer student at the University of Saint Andrews. The notable changes I made to the code-base over this period will be clearly outlined in this section to distinguish them from the work which was undertaken from September 2017 onward as part of this project, however these will be not be described in depth, in order to avoid removing focus from the more relevant implementations. If more detail on the work conducted in this summer studentship is of interest, it can be found in the slides of a presentation conducted at the 2017 St Andrews Mathematics Undergraduate Research Conference, which have been included in the appendix. 

\subsubsection{Object Oriented Additions}
The R programming language allows for packages to include their own classes to produce objects with behaviours and properties tailored to the task at hand. These typically come in the form of S3 or S4 objects. As part of my summer work I wrote methods for S3 generic functions which allow the LT2D package to be more streamlined and provide the user with a cleaner and more straight forward interface with the functionality provided under the hood. The first set of these methods are dedicated to the R plot function, allowing the user to plot LT2D fitted model objects with a simple call to the generic plot function, with the aforementioned model object as the first argument. 

\subsubsection{Adding in Abundance Estimation}
Previous versions of the code-base only produced estimates for average detection probability, $\hat{p}$. Code supplied by Prof. Borchers was modified and integrated with existing LT2D code to allow the package to produce estimates of abundance as well as average detection probability and effective strip half width. This work was done towards the end of the project and hence was not fully integrated and tested with all other pieces of functionality available within the software, therefore modifications of abundance estimation code and its continual testing and iterative improvement is necessary for the completion of this project, however the details of these updates are of far less interest than the more ambitious aspects of the project and so they shall not be mentioned further in any significant amount of detail. Any addition or modification of code or theoretical technicalities related to the LT2D package mentioned out-with of section 1.2 are not a result of previous work and form the improvements and extensions which the project aims to produce.

\subsubsection{Rounded Angles}
When measurements of small detection angles are rounded down to $0$ in the field, the magnitude of the bias of estimated average detection probability, $\hat{p}$, can drastically increase. An appropriate modification to the method's likelihood function was theorised by prof. Borchers and then I implemented the new functionality into the source code. I conducted multiple simulations to confirm that bias was significantly reduced, all of which clearly demonstrated the new estimator performed far better.

\subsubsection{New Perpendicular Density}
A reparameterisation of an existing perpendicular density was required to allow for a  half-normal perpendicular density function with a non-zero expected frequency of animals on the transect line. Prof. Borchers and I decided on a suitable formulation and I implemented it within the LT2D package. 

\subsubsection{Concatenation of GoF Functions}
Goodness of fit tests (Cramer Von Mises and Kolmogorov-Smirnov) are conducted separately in the x and y (perpendicular and forward) directions, respectively, by the LT2D. These were split into distinct functions, making their use tedious to the end user. I rectified this issue by creating a wrapper function which was easier for the user to interact with and acted as an interface with the pre-existing code.

\subsubsection{Model Collecting and Summarising}
As with many model-based statistical methodologies, it is often good practice to fit a range of varied models to the available data, and to use selection criteria (such as Akaike's Information Criterion) to select which models are most reasonable. This leads to the user's R global environment being peppered with objects created by the LT2D package, making it difficult to keep track of the models of interest. Inspired by another piece of software used for animal abundance (RMARK - An interface to the popular MARK windows software used as a tool for analysis of mark-recapture data) - model-collecting and comparison-table-generating functions were implemented within the LT2D package.

\subsection{Project objectives and outline}
The project aims to include the ability for users to specify covariate inclusion for the two-dimensional detection functions, making them not only a function of perpendicular and forward distance, but also of user-defined relationships between other variables. Upon discussion with Prof. Borchers, it was decided this functionality should be kept as general as possible, following the following main criteria:

\begin{enumerate}
\item Users should be able to choose whether the covariate affects the behaviour of the detection function in the perpendicular direction, the forward direction or at the intercept, or any combination of the above.
\item Users should be able to define the formula which describes the linear predictor for the relevant parameter and the covariate(s) of interest themselves, rather than be restricted to pre-defined options
\item Users should be able to specify different formulas for each parameter (forward, perpendicular and intercept) wherever the detection function allows these to be different.
\item There should be some level of inbuilt error-checking to ensure that users choices for formulas and starting parameter values are consistent between themselves and their choice of detection function
\end{enumerate}

All code calculating estimates of abundance ($\hat{N}$), average detection probability ($\hat{p}$) and AIC should be modified to work with all reasonable choices of covariates and related linear predictors. S3 methods should be written for the plot function to allow it to handle fitted models which contain cases of both discrete and continuous covariate inclusion. The goodness of fit function should also be generalised to use the Kolmogorov-Smirnov and Cramer Von Mises tests when the fitted model has included covariates. The inclusion of code which produces (non-parametric) bootstrap confidence intervals of estimated abundance would also be desirable, for both covariate-including and covariate-excluding models.

\section{Implementation}
\subsection{Covariates; why include them, and how?}
There are many reasons for the desirability of covariate inclusion within the LT2D package. Firstly, when dealing with animal populations which are heterogeneous with respect to detection probability (that is, individuals in the population of interest do not all share a common detection function due to their varying behaviours or characteristics), failing to take this into account often leads to biased estimates of abundance and can even lead to an inability to distinguish amongst reasonable models \cite{Link}, this can be especially prominent in Mark-Recapture analyses, which are the subject of the aforementioned citation. Although distance sampling benefits from the property of pooling robustness, and hence can cope with reasonable levels of heterogeneity within a population \cite{Buckland2004}, 2D detection sampling is based more fundamentally on time-to-detection models and survival analysis, meaning there is no theoretical guarantee that it also shares this robustness with respect to heterogeneous populations. Secondly, we may be interested in performing an analysis on data for which we have associated environmental factors of particular interest. By fitting various models which include different combinations of these covariates, analysing these models' estimated parameters, and by comparing them by standard means of model selection and goodness-of-fit, we may gain insight into the relationship between these covariates and the detection probabilities of individuals within the survey population. Finally, we hope to be able to reduce variance in estimates of abundance and density, by providing more information with which to assess detection probability than the coordinates of detected individuals alone.

We examine the detection functions outlined in \cite{Borchers} and available as part of the associated code-base in order to gain an appreciation for their general form and to ascertain which parameters are most appropriate for covariate inclusion. It should be noted that these detection functions are formulated as hazards of detection, given the location of an animal. Therefore, the terms 'detection function' and 'hazard function' will be used interchangeably henceforth, to allow us to remain consistent with the previous code-base of the LT2D package. Although other, less used hazard functions are present within this code-base, they shall not be examined nor considered for covariate implementation.

%h1
\begingroup
\large

\begin{equation}
h_1\left(y,x;\boldsymbol{\theta}\right) = \theta_1\left(y^2+x^2\right)^{-\frac{\theta_2}{2}}
\end{equation}

%ip1
\begin{equation}
ip_1\left(y,x;\boldsymbol{\theta}\right) = \theta_1\left\{\frac{1}{\sqrt{1+\left(\frac{x}{\theta_2}\right)^2+\left(\frac{y}{\theta_2}\right)^2}}\right\}^{\theta_3 + 1}
\end{equation}

%ip2
\begin{equation}
ip_2\left(y,x;\boldsymbol{\theta}\right) = \theta_1\left\{\frac{1}{\sqrt{1+\left(\frac{x}{\theta_2}\right)^2+\left(\frac{y}{\theta_4}\right)^2}}\right\}^{\theta_3 + 1}
\end{equation}

%ep1

\begin{equation}
ep_1\left(y,x;\boldsymbol{\theta}\right) = \theta_1\left\{\left(\frac{x}{\theta_2}\right)^{\theta_3}+\left(\frac{y}{\theta_2}\right)^{\theta_3}\right\}
\end{equation}

%ep2$
\begin{equation}
ep_2\left(y,x;\boldsymbol{\theta}\right) = \theta_1\left\{\left(\frac{x}{\theta_2}\right)^{\theta_3}+\left(\frac{y}{\theta_4}\right)^{\theta_3}\right\}
\end{equation}
\endgroup

It should be noted that we introduce a log link function to all $\theta$ parameters, in order to allow numerical optimisation routines tasked with finding the maximum likelihood estimates of these parameters the possibility to perform their search anywhere on the real axis. This is with the exception of $ep_1$ and $ep_2$, which use a logit link function for their $\theta_1$ parameters instead.

We notice that all of the above hazard functions share a similar parameter structure. Firstly, they all contain a scale parameter which multiplies the whole function. This is $\theta_1$, which can be interpreted as an 'intercept' for detection probability, given an increase in its value will increase the detection function's value for a given $x$ and $y$, and a reduction in its value will cause the opposite. Secondly, all hazard functions contain parameters which affect the behaviour of the function in the $x$ and $y$ directions. Sometimes the behaviour in both these directions is determined by a single parameter (such as $h_1$, $ip_1$ and $ep_1$, which have both their $x$ and $y$ direction dependent on $\theta_2$), however, in the cases of $ip_2$ and $ep_2$, the inclusion of a $\theta_4$ parameter allows the behaviour in the $y$ direction to be different to that in the $x$ direction.

In the case of $\theta_1$, modifying this parameter to include covariates would enable us to allow for heterogeneity which we believe affects an individual's detection probability, but not the shape of its detection function. In the case of $\theta_2$, covariate inclusion would cause changes in detection function shape in either the $x$ direction alone, or in the $x$ and $y$ directions in the same way, depending on the user's selection of hazard within their model criteria. Any change in $\theta_4$ would cause a change in detection function shape in the $y$ direction alone. We propose the following reformulations of $\theta_1$, $\theta_2$ and $\theta_4$ which allow them to be modelled as the combination of an intercept and a user-specified formula:

\begingroup
\large
\begin{equation}
\mathcal{G}_m\left(\theta_1\right) = \lambda_1 + \mathcal{C}_i\left(\boldsymbol{\gamma}_i\right)
\end{equation}

\begin{equation}
ln\left(\theta_2\right) = \lambda_2 + \mathcal{C}_x\left(\boldsymbol{\gamma}_x\right)
\end{equation}

\begin{equation}
ln\left(\theta_4\right) = \lambda_4 + \mathcal{C}_y\left(\boldsymbol{\gamma}_y\right)
\end{equation}
\endgroup

Where $\mathcal{G}_m$ is the link function appropriate to $\theta_1$ given the detection function given by model specification $m$ the user has selected, $\lambda_1$, $\lambda_2$ and $\lambda_4$ are intercept parameters for the $\theta$ parameter with the corresponding subscript, $\mathcal{C}_i$ is the user specified formula for covariate inclusion in the detection function's intercept, $\mathcal{C}_x$ is the user specified formula for covariate inclusion in the $x$ direction, and $\mathcal{C}_y$ is the user specified formula for covariate inclusion in the $y$ direction. The $\boldsymbol{\gamma}$ are vectors of parameters of the appropriate length required for the corresponding covariate relationship, $\mathcal{C}$. We must have $\mathcal{C}_x=\mathcal{C}_y$ if the detection function specified within model $m$ is $h_1$, $ip_1$ or $ep1$, since $\theta_2$ will affect the detection function's shape in both the $x$ and $y$ directions.

Before considering an example model specification, we introduce the perpendicular density functions within the LT2D package which allow the methodology to account for (and describe) the nature of the observed responsive movement:

\begingroup
\large

\begin{equation}
\pi_U\left(x;\boldsymbol{\phi}\right)=\frac{1}{w}
\end{equation}

\begin{multline}
\pi_{HN}{\left(x;\boldsymbol{\phi}\right)}= e^{-\frac{x^2}{2\phi_1^2}}\left\{\int_0^w{e^{-{\frac{x^2}{2\phi_1^2}}}\ \mathrm{d}x}\right\}^{-1}
\end{multline}

\begin{multline}
\pi_N\left(x;\boldsymbol{\phi}\right)=\\e^{-\frac{\left(x-\phi_2\right)^2}{2\phi_1^2}}\left\{\int_0^w{e^{-\frac{\left(x-\phi_2\right)^2}{2\phi_1^2}}\mathrm{d}x}\right\}^{-1}
\end{multline}

\begin{multline}
\pi_{CN}\left(x;\boldsymbol{\phi}\right) =\\ \left\{1-e^{-\frac{x^2}{2\phi_1^2}}\right\}\left\{w - \int_0^w{e^{-{\frac{x^2}{2\phi_1^2}}}\ \mathrm{d}x}\right\}^{-1}
\end{multline}

\begin{multline}
\pi_{CNi}\left(x;\boldsymbol{\phi}\right) = \\ \left\{1-\phi_2e^{-\frac{x^2}{2\phi_1^2}}\right\}\left\{w - \phi_2\int_0^w{e^{-{\frac{x^2}{2\phi_1^2}}}\ \mathrm{d}x}\right\}^{-1}
\end{multline}
\endgroup

$\pi_U$ reflects the assumption of no responsive movement, whereas $\pi_{HN}$ reflects the assumption that animals we be attracted to the transect line and $\pi_{CN}$ reflects the assumption that animals will display avoidance. $\pi_N$ is flexible enough to model both attraction and avoidance. \cite{Borchers}. $\pi_{CNi}$ is a reparameterisation of $\pi_{CN}$ which allows for non-zero expected abundance on the transect line.

For illustration, we now consider an example model $m$ specified by a user of the LT2D package, which selects an $ip_2$ detection function as well as a $\pi_{HN}$ perpendicular density distribution. Data collected by the user includes all of the standard information required to perform an estimate of abundance through conventional distance sampling (CDS), as well as the forward distance of each detection, a factor covariate $species$ (with three levels), a continuous covariate $forest.cover$ and another continuous covariate $size$, which for each detection refers to the number of individuals in the detected group. The user has specified that the $i$ direction (the intercept) should include the relationship $i\sim forest.cover$ (using R tilde notation for the specification of formulas). The user would also like to assume that $x\sim species$ and $y\sim size$ to specify the nature of covariate inclusion in the $x$ and $y$ directions respectively. Given this, the full formula for the detection function is given by:

\begin{multline}
hr_m\left(y,x;\boldsymbol{\theta}\right) = \theta_1\left\{\frac{1}{\sqrt{1+\left(\frac{x}{\theta_2}\right)^2+\left(\frac{y}{\theta_4}\right)^2}}\right\}^{\theta_3 + 1}\\\theta_1=exp\left\{{\lambda_1 + \gamma_{1,1}forest.cover_{y,x}}\right\} \\\theta_2 = exp\{\lambda_2 + \\\mathcal{I}_2\left(species_{y,x}\right)\gamma_{2,1}\\ +\mathcal{I}_3\left(species_{y,x}\right)\gamma_{2,2}\}  \\\theta_3 = exp\left\{\lambda_3 + \gamma_{3,1}size_{y,x}\right\}
\end{multline}

With equation (13) perpendicular density function. Where $hr_m$ represents the hazard detection function corresponding to the model specification m and the $\lambda$ parameters are intercepts for the corresponding $\theta$ parameters. $forest.cover_{y,x}$, $species_{y,x}$, and $size_{y,x}$ are the values of $forest.cover$, $species$ and $size$ respectively, for the detection at coordinate $(x,y)$.

It becomes clear that allowing the user a selection of choices for covariate inclusion with complete freedom of choice for formulas as well as which detections function to select, which perpendicular density function to select, and allowing each parameter to have a different formula quickly becomes an issue of not only statistical theory, but also a problem rooted in software design. We must construct a general methodology which allows users to specify in simple terms the model specification they desire, with software capable of producing the corresponding design matrices and linear predictors whilst also being able to error-check the user's inputs for inconsistency. For example, a user attempting to fit different covariate formulas in the $x$ and $y$ directions when hazard function $h_1$ is selected should raise an error, since this detection function is not flexible enough to allow this. The following section provides an insight into the choices made to achieve the implementation of the above described functionality. Individuals primarily concerned with learning how to use the new LT2D software to produce analyses on their own data should feel free to skip to chapter 5.

\subsection{Understanding the LT2D software and its architecture}
In order to implement covariate inclusion it is vital to first understand the inner workings of the LT2D package, so that we may ascertain both the complexity of the task at hand, and also the most efficient method with which to proceed. A simplification of the software's top-level fitting function call hierarchy (as of August 2017, before the project began), which retains the features of greatest importance and relevance is illustrated by figure 4.

\begin{figure}[H]
% the following code was used to produce the image. It was then exported as an image and then reincluded into the document as a static to allow for easier resizing
%\begin{tikzpicture}[scale=1.5]
%\Tree[.LT2D.fit [.fityx [.optim [.negloglik.yx [.p.pi.x [.pi.x ]  [.px [.Sy [.hr ] [.pi.x ] ] ] ] [.fyx [.hr ]  ] ] ] [.solve ] ][.NDest  [.invp1\_replacement [.phat [.hr ] [.pi.x ] ] ] ] ]
%\end{tikzpicture}
\centering
\includegraphics[scale=0.507]{LT2Dprecov}
\caption{Simplified call hierarchy for the top-level fitting function in the LT2D software}
\end{figure}
This diagram illustrates in more detail the inner workings of the LT2D package, compared to the specifications of the methodology laid out by \cite{Borchers}. Below a brief outline and description of each of the functions in figure 4 will be given, to give the reader a general understanding of how the LT2D package used to fit models to data before covariate implementation, and how it evaluates the likelihood function outlined in \cite{Borchers}.

\begin{itemize}
\item LT2D.fit ; A function which given an appropriate input R data.frame and model specification settings, produces a two-dimensional detection function distance sampling maximum likelihood estimation of model parameters and abundance.

\item fityx ; The function responsible for the model-fitting provided by LT2D.fit.

\item NDest ; The function tasked with calculating estimates of abundance and density, when provided with the output of the fityx function.

\item optim ; The linear optimisation routine which provides maximum likelihood estimation of the model parameters.

\item solve ; The function which solves the hessian matrix provided by optim in order to produce a variance-covariance matrix for the estimated parameters.

\item invp1\_replacement ; The function which produces an estimate of inverse effective half strip width, given the chosen model settings and their estimated parameters.

\item negloglik.yx ; The function which evaluates the negative log likelihood - the probability of observing the observations we saw, given the model specifications and their estimated parameters.

\item phat ; The function tasked with calculating effective half strip width, given model specifications and their estimated parameters.

\item p.pi.x ; The function which evaluates the product of the probability of detection at a given perpendicular distance and the probability of an animal's presence at that perpendicular distance, given model specifications and their estimated parameters.

\item fyx ; Calculates the probability density function of the 'waiting distance', given model specifications and their estimated parameters.

\item hr ; Evaluates the two-dimensional detection function specified by the user as part of the model specification. Returns the hazard rate of detection given estimated parameters.

\item pi.x ; Evaluates the assumed perpendicular density distribution of animals about the transect line, selected by the user. Returns the probability of the animal's location at a given perpendicular distance given estimated parameters.

\item px ; Numerically calculates the perpendicular detection function from a specification of a hazard function (in the form of hr, above) and its estimated parameters.

\item Sy ; Calculates the survivor function to a given forward distance. Evaluates the probability of an animal remaining undetected until a given forward distance ahead of the observer, and then being detected at that distance, given model specifications and their estimated parameters.
\end{itemize}

\subsection{Covariate Inclusion into the Likelihood}
\subsection{Covariate Inclusion into Abundance Estimation}
\subsection{Modification of Goodness of Fit Functions}
\subsection{Modification of Plot Functions}
\section{Usage of the new software}
\subsection{Model specification for the user}
\subsection{Simulation studies}

\section{Going forwards}
\subsection{Additional desired features}
\subsection{A proposed approach for mixture model incorporation}


\bibliographystyle{apacite}
\bibliography{Biblio}
\end{multicols}
\end{document}
